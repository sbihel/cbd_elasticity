\documentclass{beamer}
\usepackage[utf8]{inputenc}

\usepackage{appendixnumberbeamer}

\usetheme[sectionpage=progressbar,subsectionpage=progressbar,numbering=fraction,
          progressbar=foot]{metropolis}

\title{Cloud \& Big Data}
\subtitle{Elasticity in Cloud Computing}

\date{\today}
\author{%
  Simon Bihel, \url{simon.bihel@ens-rennes.fr} \\
  Rémi Hutin, \url{remi.hutin@ens-rennes.fr}
}
\institute{%
  University of Rennes I \\
  École normale supérieure de Rennes
}

\begin{document}

\maketitle

\begin{frame}{Table of contents}
  \setbeamertemplate{section in toc}[sections numbered]
  \tableofcontents[hideallsubsections]
\end{frame}


\section{Elasticity: Definition and Differentiation}
\begin{frame}
  \frametitle{What It Is Not~\cite{herbst2013elasticity}}
  \begin{description}
    \parbox{\linewidth}{
    \item[Scalability] Sustain increasing workloads with adequate performance
    \item[Efficiency] Amount of resources consumed for a given amount of work
    }
  \end{description}
\end{frame}


\begin{frame}
  \frametitle{What It Is Not~\cite{herbst2013elasticity}}
  \centering
  \Large\textbf{Elasticity} $\neq$ \textbf{Scalability}
\end{frame}

\begin{frame}
  \frametitle{Elasticity~\cite{herbst2013elasticity}~\cite{galante2012survey}~\cite{gulati2011cloud}~\cite{sharma2011cost}~\cite{moore2013coordinated}}
  \begin{definition}
  \parbox{\linewidth}{\textbf{Elasticity} is the degree to which a system is able to adapt to
    workload changes by provisioning and de-provisioning resources in an
    autonomic manner, such that at each point in time the available resources
    \textit{match} the current demand as closely as possible.
  }
  \end{definition}
\end{frame}

\begin{frame}
  \frametitle{Example}
  %\includegraphics{workload_elasticity}
\end{frame}


\section{When to Scale}
\begin{frame}
  \frametitle{Reaction}
  Set of rules.
\end{frame}

\begin{frame}
  \frametitle{Prediction}
\end{frame}

\begin{frame}
  \frametitle{Example}
\end{frame}


\section{How to Scale}
\begin{frame}
  \frametitle{Mechanisms}
  \begin{description}
    \item[Migration]
    \item[Resizing] \textit{Might be a migration.}
    \item[Replication]
  \end{description}
\end{frame}

\begin{frame}
  \frametitle{Configurations \& Transitions}
  \vspace*{\fill}
  There is also cost for transition.
\end{frame}

\begin{frame}
  \frametitle{Example}
\end{frame}


\section*{Conclusion}
\begin{frame}
  \frametitle{Conclusion}
\end{frame}


\appendix
\section*{References}
\begin{frame}[allowframebreaks]{References}
  \bibliography{bib}
  \bibliographystyle{abbrv}
\end{frame}


\end{document}
